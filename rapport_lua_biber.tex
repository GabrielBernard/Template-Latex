<+  +>  !comp!  !exe!
%        File: !comp!expand("%")!comp!
%     Created: !comp!strftime("%a %b %d %I:00 %p %Y ").substitute(strftime('%Z'), '\<\(\w\)\(\w*\)\>\(\W\|$\)', '\1', 'g')!comp!
% Last Change: !comp!strftime("%a %b %d %I:00 %p %Y ").substitute(strftime('%Z'), '\<\(\w\)\(\w*\)\>\(\W\|$\)', '\1', 'g')!comp!
%
\documentclass[letterpaper,12pt,oneside]{article}

\usepackage{polyglossia} % Use instead of babel for xelatex and lualatex
\usepackage{fontspec} % Use instead of fontenc for xelatex and lualatex
\setdefaultlanguage{french} % Use tu set the language for xelatex and lualatex

% Set font to use in document
\setmainfont{Times New Roman} 
% Exemple with custom ttf font
% \setmainfont{IMFePIrm29P.ttf} [
%   ItalicFont = IMFePIit29P.ttf,
%   SmallCapsFont = IMFePIsc29P.ttf,
%   StylisticSet=1,
%   WordSpace = 1.2,
%   Ligatures = Rare,
%   Style = Historic,
% ]

\usepackage[squaren,Gray]{SIunits} %for SI units recognition
\usepackage{numprint}
\usepackage[left=1in, top=1in, right=1in, bottom=1in]{geometry} %To set page margins
\usepackage{graphicx} % For figures (needs to be befor epstopdf)
\usepackage{fancyhdr} 
\usepackage{float} % To place figure with [H], [!ht]\dots
\usepackage{array}
\usepackage{csquotes}
\usepackage{booktabs} % For better tables
\usepackage{textcomp}
\usepackage{amsmath} % For advance math capability
\usepackage{amsfonts}
\usepackage{bm} % Bold math
\usepackage{multirow} %To use multi row in tables
\usepackage{epstopdf} %To use .eps figure
\usepackage{caption} % For better captions
\usepackage{parskip} % For paragraph space
%\usepackage{subcaption} %for subfigure
%\usepackage{color} %pour mettre de la couleur
%\usepackage{wrapfig} % figure funky positions
%\usepackage{listings}
%\usepackage[]{mcode}%code matlab en couleur il faut avoir le package dans le chemin de travail

\usepackage{hyperref} % Add link to pdf (Table of Content, references, fig, tables...)

% Bibliography with biber
\usepackage[
    backend=biber,
    style=ieee,
    hyperref=auto
]{biblatex}

\addbibresource{bibliographie.bib}

% Figures can be in a separate directory named figures
\graphicspath{{figures/}{../figures/}}

\newcommand{\HRule}{\rule{\linewidth}{0.5mm}}

% Change names some caption for french usage
\addto\captionsfrench{\def\tablename{\textsc{Tableau}}}
\addto\captionsfrench{\renewcommand*\abstractname{Sommaire}}
\renewcommand{\tablename}{Tableau}

\providecommand{\e}[1]{\ensuremath{\times 10^{#1}}}
\sloppy
\renewcommand\floatpagefraction{.9}
\renewcommand\topfraction{.9}
\renewcommand\bottomfraction{.9}
\renewcommand\textfraction{.1}
\setcounter{totalnumber}{50}
\setcounter{topnumber}{50}
\setcounter{bottomnumber}{50}
\geometry{hscale=0.82,vscale=0.78}

% Add an appendix and changes caption for french usage
\usepackage[toc,page]{appendix}
\renewcommand{\appendixtocname}{Annexes}
\renewcommand{\appendixpagename}{Annexes}

\begin{document}

% CONTENU DE LA PAGE TITRE %%%%%%%%%%%%%%%%%%
% Remplir avec les informations spécifiques
%%%%%%%%%%%%%%%%%%%%%%%%%%%%%%%%%%%%%%%%%%%%%

\begin{titlepage}

%\newcommand{\HRule}{\rule{\linewidth}{0.5mm}} % Defines a new command for the horizontal lines, change thickness here

\center % Center everything on the page
 
%----------------------------------------------------------------------------------------
%   HEADING SECTIONS
%----------------------------------------------------------------------------------------

\textsc{\LARGE University Name}\\[1.5cm] % Name of your university/college
\textsc{\Large Major Heading}\\%[0.5cm] % Major heading such as course name
\textsc{\large Minor Heading}\\%[0.5cm] % Minor heading such as course title
\vfill
%----------------------------------------------------------------------------------------
%   TITLE SECTION
%----------------------------------------------------------------------------------------

\HRule \\[0.4cm]
{\huge \bfseries Title}\\[0.4cm] % Title of your document
\HRule \\%[1.5cm]
\vfill 
%----------------------------------------------------------------------------------------
%   AUTHOR SECTION
%----------------------------------------------------------------------------------------

\begin{minipage}{0.4\textwidth}
\begin{flushleft} \large
\emph{Écrit par:}\\
Gabriel \textsc{Bernard} % Your name
\end{flushleft}
\end{minipage}
~
\begin{minipage}{0.4\textwidth}
\begin{flushright} \large
\emph{Présenté à:} \\
Dr. James \textsc{Smith} % Supervisor's Name
\end{flushright}
\end{minipage}\\%[4cm]
\vfill

% If you don't want a supervisor, uncomment the two lines below and remove the section above
%\Large \emph{Author:}\\
%John \textsc{Smith}\\[3cm] % Your name

%----------------------------------------------------------------------------------------
%   DATE SECTION
%----------------------------------------------------------------------------------------

{\large \today}\\[3cm] % Date, change the \today to a set date if you want to be precise

%----------------------------------------------------------------------------------------
%   LOGO SECTION
%----------------------------------------------------------------------------------------

%\includegraphics{Logo}\\[1cm] % Include a department/university logo - this will require the graphicx package
 
%----------------------------------------------------------------------------------------

%\vfill % Fill the rest of the page with whitespace

\end{titlepage}

%Insert blanck page
%\newpage
%\mbox{}
%\thispagestyle{empty}
%\newpage

%NUMÉROTER en chiffre romain les tableaux et figures
\pagenumbering{Roman}
\tableofcontents
\listoffigures
\listoftables


%Insert blanck page
%\newpage
%\mbox{}
%\thispagestyle{empty}
\newpage

\setcounter{page}{1}
\pagenumbering{arabic}


\setlength{\parindent}{15pt} %To set paragraphs space to 15pt

\begin{abstract}
\dots
\end{abstract}

\section{Introduction}
\dots

\section{Théorie}
\dots

\section{Méthodologie}
\dots

\section{Résultats}
\dots

\section{Discussion}
\dots

\section{Conclusion}
\dots

\pagebreak

\section*{Remerciements}

\pagebreak
\setlength\bibitemsep{0.5\baselineskip} % Set spacing between bibitems
\printbibliography

\pagebreak
\begin{appendices}
\pagenumbering{Roman} \setcounter{page}{1}
\begin{subappendices}

\end{subappendices}
\end{appendices}
\end{document}
