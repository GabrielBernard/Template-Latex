<+	+>	!comp!	!exe!
%        File: !comp!expand("%")!comp!
%     Created: !comp!strftime("%a %b %d %I:00 %p %Y ").substitute(strftime('%Z'), '\<\(\w\)\(\w*\)\>\(\W\|$\)', '\1', 'g')!comp!
% Last Change: !comp!strftime("%a %b %d %I:00 %p %Y ").substitute(strftime('%Z'), '\<\(\w\)\(\w*\)\>\(\W\|$\)', '\1', 'g')!comp!
%
\documentclass[letterpaper,12pt,oneside]{article}
\usepackage[french]{babel}
\usepackage[utf8]{inputenc}
\usepackage[T1]{fontenc}
\usepackage[squaren,Gray]{SIunits}
\usepackage{numprint}
\usepackage[left=1in, top=1in, right=1in, bottom=1in]{geometry}
\usepackage{graphicx}
\usepackage{fancyhdr}
\usepackage{titling}
\usepackage{float}
\usepackage{comment}
\usepackage{url} 
\usepackage{array}
\usepackage{changepage}
\usepackage{csquotes}
\usepackage{booktabs}
\usepackage{textcomp} 
\usepackage{amsmath}
\usepackage{amsfonts}
\usepackage{lscape}
\usepackage{multirow}
\usepackage{epstopdf}
\usepackage{caption}
\usepackage{subcaption} %for subfigure
%\usepackage[a4paper]{geometry}
%\usepackage{color} %pour mettre de la couleur
%\usepackage{wrapfig} % figure funky positions 
%\usepackage{listings}
%\usepackage[]{mcode}%code matlab en couleur il faut avoir le package dans le chemin de travail
\graphicspath{{images/}{../images/}}

\newcommand{\HRule}{\rule{\linewidth}{0.5mm}}
\addto\captionsfrench{\def\tablename{\textsc{Tableau}}}
\addto\captionsfrench{\renewcommand*\abstractname{Sommaire}}
\renewcommand{\tablename}{Tableau}
\providecommand{\e}[1]{\ensuremath{\times 10^{#1}}}
\sloppy
\renewcommand\floatpagefraction{.9}
\renewcommand\topfraction{.9}
\renewcommand\bottomfraction{.9}
\renewcommand\textfraction{.1}   
\setcounter{totalnumber}{50}
\setcounter{topnumber}{50}
\setcounter{bottomnumber}{50}
\geometry{hscale=0.82,vscale=0.78}
\usepackage[toc,page]{appendix}
\renewcommand{\appendixtocname}{Annexes}
\renewcommand{\appendixpagename}{Annexes}
\usepackage{booktabs}


% CODE MATLAB
%\usepackage{listings}



\begin{document}

% CONTENU DE LA PAGE TITRE %%%%%%%%%%%%%%%%%%
% Remplir avec les informations spécifiques
% à chaque labo
%%%%%%%%%%%%%%%%%%%%%%%%%%%%%%%%%%%%%%%%%%%%%

\newcommand{\noms}{Nom et matricule}
\newcommand{\cours}{cours}
\newcommand{\quoi}{Sujet}
\newcommand{\titre}{Titre}
\newcommand{\prof}{Prof}

\begin{titlepage}

\begin{center}

\textsc{\LARGE \cours}
\\[1.5cm]
\textsc{\Large \quoi}\\[0.8cm]

\vfill
% Title
\HRule \\[0.4cm]
{ \huge \bfseries \titre}\\[0.1cm]
\HRule

\vfill
 
\Large{Présenté à \\ \prof}\\[2cm]

\vfill

\Large{Par\\ \noms}\\[0.75cm]

\vfill


\includegraphics[width=6.3145cm,height=3cm]{polytechnique.eps} 
\hspace{1cm}
\includegraphics[width=3cm,height=3cm]{LOGO_GP.png}
\\
\today



\end{center}
\end{titlepage}

%Insert blanck page
\newpage
\mbox{}
\thispagestyle{empty}
\newpage

\pagenumbering{Roman}
\tableofcontents
\listoffigures
\listoftables
%NUMÉROTER en chiffre romain les tableaux et figures

%Insert blanck page
\newpage
\mbox{}
\thispagestyle{empty}
\newpage

\setcounter{page}{1}
\pagenumbering{arabic}

\begin{abstract}

\end{abstract}

\section{Introduction}


\section{Théorie}

\section{Méthodologie}

\section{Présentation et analyse des résultats}

\section{Sources d'erreurs}


\section{Conclusion}


\subsection*{Remerciements}

\bibliographystyle{ieeetr}
\bibliography{biblio1}

\pagebreak
\begin{appendices}
\pagenumbering{Roman} \setcounter{page}{1}
\begin{subappendices}

\end{subappendices}
\end{appendices}
\end{document}
